\documentclass[a4paper,12pt,twocolumn]{article}
\usepackage[spanish]{babel}
\usepackage[left=2.5cm, right=2.5cm, top=3cm, bottom=3cm]{geometry}
\usepackage{amsmath, amsthm, amssymb}
\usepackage{minted}


\begin{document}
\title{Moogle!}
\author{Jocdan L. López Mantecón \\ Facultad de Matemáticas y Computación, Universidad de la Habana}
\date{Julio \\ 2023}
\maketitle

\begin{abstract}
El objetivo de este proyecto es crear un motor de búsqueda de documentos de archivos txt
basándose en el Modelo del Espacio Vectorial para la Recuperación de Información. Se realizó
una revisión bibliográfica previa de la cual se decidieron los siguientes procedimientos: Cada
documento se representa como un vector de sus términos en el Corpus de términos, constituido
por todas las palabras de la colección de documentos. Como medida de la importancia de las
palabras, se recurrió al factor TF-IDF. Así mismo, se trata la query como un pseudo-documento
representado por su propio vector de términos del Corpus. Para la determinación de la similitud
entre la query introducida por el usuario y los documentos de la colección se utilizó el indicador
Similitud de Cosenos que da una medida de la distancia entre el vector query y los vectores-documento
a partir de la amplitud del ángulo que forman estos dos vectores en el espacio
vectorial de vectores-documento. Además, se generó un snippet para cada resultado de la
búsqueda y se implementaron cuatro operadores de búsqueda que facilitan el control sobre la
misma.
Se analizaron estructuras de datos factibles para la ejecución del mismo donde
resaltan, además de los conocidos array<T>, las colecciones List<T> y HashSet<T>
destacándose este último por su elevado rendimiento temporal. También se estudió el tipo
System.String y sus métodos para la generación de Snippets y las expresiones lambda
\end{abstract}

\section*{Objetivos}
\begin{enumerate}
    \item Crear un motor de búsqueda de documentos en formato .txt empleando el Modelo de
    \item Espacios Vectoriales para la Recuperación de Información.
    \item Generar un snippet para cada resultado de la búsqueda.
    \item Implementar utilidades adicionales que mejoren la calidad de la búsqueda y el
    control sobre la misma.
    \end{enumerate}

\section{Introducción}\label{sec:intro}

\subsection{Modelo de Espacio Vectorial para la Recuperación de Información (MEV)}
El modelo de espacio vectorial es una técnica estándar en la los Sistemas de Recuperación de
Información, consiste en representar los documentos como vectores de las palabras que los
constituyen. Se construye un vocabulario con todas las palabras de la colección de documentos
de forma que no se repitan los términos y cada uno de estos términos representa una dimensión
del espacio. Los vectores-documento representaran la importancia de cada término del
vocabulario en el propio documento mediante su valor de peso, así mismo la query se trata
como un pseudo-documento y se representa mediante un vector en el espacio. Luego, se
emplean indicadores de similitud para determinar que documentos son similares a otros o a una
query dada.

\subsection{Term Frequency Inverse Document Frequency (TF-IDF):}
Esta medida de peso de los términos se basa en dos medidas estadísticas:
\begin{itemize}
    \item \emph{Frecuencia de términos (TF):} mide la frecuencia relativa con la que un término aparece
    en un documento, es decir, cuantas veces aparece el término en el documento
    (frecuencia absoluta Nw) dividido por la cantidad total (Td) de palabras del documento. La
    fórmula utilizada es:
    
    \begin{center}    
    \begin{equation}
        TF_w = \frac{N_w}{T_d}
    \end{equation}
    \end{center}
    La intuición con esta medida es que los términos que aparecen más veces
    probablemente sean más importantes para el contenido del documento.

    \item \emph{Frecuencia inversa de documento (IDF):} Mide la importancia del término en la colección
    de documentos, es decir, cuantos documentos de la colección contienen el término.
    Se empleó la fórmula:
        \begin{center}
        \begin{equation}
         IDF_w = \log_2(\frac{|D|}{|d_w|})   
        \end{equation}
        \end{center}

    Dónde |D| es la cardinalidad de la colección de documentos, es decir, la cantidad total
    de documentos en la colección y |dw| es la cantidad de documentos de la colección que
    contienen el término. El logaritmo se utiliza para evitar asignar demasiado peso 
    a los documentos que aparecen en muchos documentos y para que los términos que aparecen
    en pocos documentos tengan un mayor peso.
    La idea que subyace esta medida es que los términos que aparecen en muchos
    documentos probablemente sean menos importantes y los que aparecen en menos
    documentos son más específicos y más relevantes para la búsqueda
\end{itemize}

La medida TF-IDF asigna un peso a cada término en cada documento de la colección. El peso
de un término se calcula multiplicando la frecuencia del término TF por la frecuencia inversa del
documento IDF:
    \begin{center}
    \begin{equation}
        weight = TF_w * IDF_w 
    \end{equation}
    \end{center}

Una vez que se han calculado los pesos para cada término en cada documento de la colección
se pueden utilizar para calcular la similitud entre una query y los documentos de la colección.
En este caso se representan los documentos como un vector de pesos y se calcula la similitud
de cosenos.

\subsection{Similitud de Cosenos:}
Es una técnica utilizada en la Recuperación de Información para comparar la similitud entre dos
vectores, se basa en el ángulo entre los mismos. Cuanto menor sea el ángulo, mayor será la
similitud entre los vectores. La similitud de cosenos se calcula como el coseno del ángulo entre
los vectores mediante la fórmula:
    \begin{center}
    \begin{equation}
        sim(A, B) = \frac{\overrightarrow{A} * \overrightarrow{B}}{||A|| * ||B||} 
    \end{equation}
    \end{center}

Donde A y B son los vectores que se están comparando, * representa el producto punto de los
vectores, y ||A||, ||B|| representan las magnitudes de los mismos.
El producto punto entre dos vectores es la suma de los productos de los componentes
correspondientes de los vectores. La magnitud de un vector es la raíz cuadrada de la suma de
los cuadrados de sus componentes. La similitud de cosenos se puede calcular directamente a
partir de los pesos de los términos en los vectores.
    
    \begin{center}
    \begin{equation}
        \overrightarrow{A} * \overrightarrow{B} = \Sigma(a_i * b_i)
    \tag*{Producto Vectorial}
    \end{equation}
    \end{center}
    
    \begin{center}
    \begin{equation}
    \tag{Norma de un vector}
        ||A|| = \sqrt{\Sigma a_i^2}
    \end{equation}
    \end{center}

\subsection{Generando el snippet:}

Para generar del snippet se decidió mostrar un subtexto del documento que contenga la palabra
de la query de mayor importancia (mayor TF-IDF). b Se extrae inicialmente un snippet con un
tamaño de 80 caracteres alrededor de la palabra mas importante de la query con respecto al
documento y al corpus; a partir de este primer resultado seguimos tomando caracteres hasta
llegar al comienzo y al final de la oración completa respectivamente buscando la primera
aparición de los signos de puntuación más comunes que pueden delimitar una oración: “ ¿? .
; ¡! ”

\subsection{Operadores de Búsqueda:}
Los operadores de búsqueda añaden opciones y control sobre los resultados de la búsqueda.
Se escriben inmediatamente antes de la palabra sobre la cual se quiere aplicar el operador.
Para su implementación se calcula un factor al que se nombró scoreModifier (sM) el cual es un
modificador porcentual que multiplicará al valor de Similitud de Cosenos entre el vector
documento y el vector query obteniéndose el resultado de Score final. Cada operador contribuye
(si aparece en la query) al valor final del scoreModifier.
    \begin{center}
    \begin{equation}
       sM = 1.00 + \^ 𝒎𝒐𝒅 + !𝒎𝒐𝒅 +  *𝒎𝒐𝒅 + ~𝒎𝒐𝒅 
    \end{equation}
    \end{center}

Se implementaron cuatro operadores de búsqueda:
\begin{enumerate}
    \item  Existe ^ : Las palabras a la que se aplica a este operador debe aparecer en los
    documentos que se muestran como resultados de la búsqueda. Si el documento a
    evaluar no contine la palabra-operando (a las cuales se denominó markers) se
    disminuye su score en un 100\%, en caso contrario permanece inalterado:

    \item NoExiste ! : Los marcadores de este operador no deben aparecer en los resultados de
    búsqueda. Si el documento contiene dicho marcador disminuye su score en 100\% en
    caso contrario permanece inalterado:

    \item Importancia *: Este operador se puede emplear N veces consecutivas al inicio de la
    palabra. Por cada vez que se utilice se incrementa el score del documento que contenga
    los marcadores de este operador en un 35\%.
        \begin{center}
        \begin{equation}
            *𝒎𝒐𝒅 = N * 0.35
        \end{equation}
        \end{center}

    \item Distancia~: Este es el único operador binario, se emplea para indicar dos palabras, la
    distancia entre estas incrementa el score del documento de forma tal que a menor
    distancia mayor es el incremento. Para utilizarlo escriba primero la primera palabra y
    luego la segunda comenzando con el operador (Ej.: Harry ~vacaciones).
    Se tuvo en cuenta que el tamaño de los distintos documentos difiere y esto podría sesgar
    dicha medida de la distancia, por ello luego de calcular la distancia entre los términos en
    caracteres, se normaliza la misma dividiéndola entre la cantidad total de caracteres del
    documento. La función que caracteriza este operador es:
        \begin{center}
        \begin{equation}
            ~𝒎𝒐𝒅 = 1.00 - \frac{dist(w_1,w_2)}{doc.Length}
        \end{equation}
        \end{center}
    Donde $dist(w_1, w_2)$ es la menor distancia entre las dos palabras en el
    documento y doc.Lenght es la cantidad total de caracteres del documento.
    De esta forma, si la distancia entre las palabras es pequeña entonces el segundo término
    de la expresión tiende a cero y el factor tiende a 1.00 (100\%); de lo contrario, si la
    distancia es grande entonces el segundo término tiende a 1 y el factor tiende a 0%.

\end{enumerate} 

\subsection{Sugerencias:}
Se implementó la funcionalidad Sugerencia basado en la Distancia de Edición o Distancia de
Levenshtein. Si alguna palabra de la query no aparece en el vocabulario del corpus y no es una
stopword entonces se calcula la distancia de edición entre dicha palabra y todos los términos
del vocabulario y sustituimos esta con la palabra del vocabulario con menor distancia de edición.
Luego se sustituye esta palabra en la query.
\subsection{Distancia de Levenshtein o distancia de Edición}
La Distancia de Levenshtein o distancia de Edición es una medida de que tan similares son dos
strings. Se define como el mínimo número de transformaciones (insertar, eliminar o remplazar)
que se necesitan para convertir uno de los strings en el otro.

\subsection{Especifiaciones de Software}
EL proyecto se desarrolló en el Sistema Operativo Windows 10 Pro y fue codificado en el editor
de texto Visual Studio Code empleando C# como lenguaje de programación y el framework
.NET SDK 7. 

\subsection{Ejecutando el proyecto}
\begin{enumerate}   
\item Asegúrese de tener instalado .NET Core 7.
\item Ir al directorio del proyecto “Moogle Project”.
\item Copiar la base de datos (colección de documentos en formato .txt) a la carpeta Content
\item Ejecutar en la terminal de Linux:
        make dev
    Si estás en Windows, desde la carpeta raíz del proyecto:
    dotnet watch run --project MoogleServer
\end{enumerate}


\section{Clases Empleadas}\label{sec: classes}
\subsection{Corpus:}
 Se creó la clase Corpus, la misma representa la colección de documentos a través
del conjunto de términos(palabras) que la componen y contiene al sistema de vectoresdocumento.
\emph{Elementos de la clase:}
\begin{enumerate}
     
    \item string[] documents;
        Contiene un listado de todos los archivos txt que se encuentran en la carpeta Content.
    \item Dictionary<string, double> idfs;
        Este diccionario le asigna a cada palabra del corpus su valor de IDF.
    \item public Dictionary<string, double> IDFs { get{return vocabulary;}}
        Propiedad que regula el acceso al campo idfs.
    \item DocumentVector[] vectorList;
        Arreglo que contiene todos los documentos de la colección representado como vectores.
    \item HashSet<string> vocabulary;
        Conjunto que contiene todas las palabras de la colección. Se emplea un HashSet para
        evitar repeticiones de términos y por su elevado rendimiento temporal en comparación
        con otras estructuras de datos.
    \item public HashSet<string> Vocabulary { get{ return words;}}
        Propiedad que regula el acceso a la variable al campo vocabulary.
    \item public Corpus(string ContentPath){…}
        Constructor de la clase Corpus, recibe como argumento la dirección de la carpeta donde
        se almacenan los archivos txt. Procesa todos los documentos convirtiéndolos en vectores
        a la vez que conforma el vocabulario, los IDFs y asigna los pesos a cada termino en cada
        documento.
    \item public void RankDocumentsBySimilarity(DocumentVector query){…}
        Método que recibe una query y le asigna a cada vector-documento un valor de Score
        determinado por la Similitud de Cosenos. Este método se emplea cuando la query no
        tiene operadores de búsqueda.
    \item public void RankDocumentsWithOperators(DocumentVector query){…}
        Este método se emplea para asignar el valor de Score a cada documento cuando la
        query emplea operadores de búsqueda. Para cada documento aplica la función de los
        operadores de búsqueda correspondientes lo que devuelve un factor que modifica el
        score (dado por similitud de cosenos) entre el documento y la query.
    \item public DocumentVector[] Ranking { get;}
        Propiedad que devuelve un arreglo de vectores-documentos organizados en orden
        decreciente, según su valor de Score, constituido por todos los documentos cuyo score
        sea mayor que 0.
    \item public HashSet<string> stopWords { get;}
        Conjunto de las stopWords. Se decidió como stopword aquellas palabras aparezcan en
        más del 85\% de los documentos.
\end{enumerate}


\subsection{DocumentVector:}
Esta clase representa a cada documento como un vector de sus términos
a la vez que contiene información útil del documento y métodos necesarios en la concepción
del modelo.
\emph{Elementos de la clase:}
\begin{enumerate}
    \item public string FileName {get; set; }
    Contiene el título del documento.
    \item public string FileText {get; private set; }
    Contiene el texto del documento.
    \item public string[] Terms{ get;}
    Vector de todas las palabras del documento (con repeticiones). Se emplea para conocer el
    tamaño del documento en palabras, la cantidad de repeticiones de cada palabra y en otros puntos
    del proyecto como el snippet y los operadores.
    \item HashSet<string> words;
    Conjunto de todas las palabras del documento (sin repeticiones)
    \item public HashSet<string> Words { get { return this.words; } }
    Propiedad que regula el acceso al campo words.
    \item Dictionary<string, int> docTermFrequency;
    Diccionario que a cada palabra del documento le asigna su valor de frecuencia relativa.
    \item double[] weights;
    Representa el vector de coordenadas del documento en el espacio de los términos del corpus
    donde el peso esta dado por el factor TF-IDF y cada término es una dimensión del espacio.
    \item public double Score { get; set; }
    Propiedad que contiene el valor de Score del documento con respecto a la query dada por la
    Similitud de Cosenos.
    \item double magnitude;
    Norma del vector de pesos.
    \item HashSet<int> nonZeroIndexes;
    Almacena los índices del vector con valor distinto de 0 para optimizar los cálculos.11. Constructor de la clase 
    DocumentVector, el texto del documento como parámetro. Inicializa y da
    valores a otros campos de la clase.
    \item public DocumentVector(string docText) {}
    Constructor de la clase DocumentVector, el texto del documento como parámetro. Inicializa y da
    valores a otros campos de la clase.
    \item public void SetWeightsInCorpus(HashSet<string> corpusWords,Dictionary<string, double> idfs) {}
     Recibe el vocabulario del corpus con los idf de todos los términos, inicializa el array weights que
    expresa al documento como un vector donde cada coordenada es un término del corpus y cada
    valor es el peso dado por tf-idf.
    \item public void Normalize() {}
    Normaliza el vector dividiendo los valores por la norma.
    \item public static double Similarity(DocumentVector v1, DocumentVector v2){}
    Aplica la fórmula de Similitud de Cosenos entre dos vectores.
    \item public static string[] Tokenize(string text) {}
    Método para extraer los tokens del texto. Divide el texto en palabras depurando todos los
    caracteres que no sean letras o dígitos y devuelve el array de palabras.
\end{enumerate}
    

\subsection{Snippet}
Snippet: Clase estática que implementa un grupo de funciones para generar el
snippet.
\emph{Elementos de la clase:}
\begin{enumerate}
    \item static string documentText;
    Contiene el texto del documento.
    \item static string lowerText;
    Aplica el método ToLower() al texto.
    \item public static string GetSnippet(DocumentVector queryVector,
    DocumentVector docVector){}
    Método principal de la clase. Recibe el vector de la query y el del documento y
    genera un snippet que muestra un subtexto del documento que contiene la palabra
    de mayor relevancia de la query para el texto.
    \item static string MostRelevantWord(DocumentVector queryVector,
    DocumentVector docVector)
    Método de que determina la palabra de mayor relevancia de la query. Empleando
    operaciones entre conjuntos, específicamente la intersección, interseca las palabras de la query, 
    el documento y el vocabulario (este último para eliminar
    stopwords) y de las palabras resultantes determina la de mayor peso (TF*IDF).
    \item static string GetTextPieceAround(int index)
    Este método extrae un subtexto alrededor de la primera aparición en el texto de la
    palabra determinada en el paso anterior, de un tamaño mayor o igual a 80
    caracteres.
    \item public static int GetIndexOf(string word)
    Determina el primer índice donde aparece la palabra más relevante.
    \item static int SentenceBeginning(int start)
    Determina el inicio de una oración a partir de los signos de puntuación que pueden
    delimitar la misma.
    \item static int SentenceEnding(int start)
    Determina el fin de una oración a partir de los signos de puntuación que pueden
    delimitar la misma.
    \item static string MyTrim(string text)
    Método que elimina los espacios en blanco innecesarios en el snippet.
\end{enumerate}

\subsection{SearchOperators}
SearchOperators: Clase estática que define e implementa los operadores de
búsqueda.
\emph{Elementos de la clase:}
\begin{enumerate}
    \item static HashSet<char> operators = new HashSet<char>("~!^*");
    Define el conjunto de símbolos que representan los operadores.
    \item static HashSet<(string word, int count)> importanceMarkers;
    Contiene las palabras que constituyen operandos del operador Importancia (*).
    \item static HashSet<string> existenceMarkers;
    Contiene las palabras que constituyen operandos del operador Existe (^).
    \item static HashSet<string> nonExistenceMarkers;
     Contiene las palabras que constituyen operandos del operador No Existe (!).
     \item static HashSet<(string, string)> distanceMarkers;
    Contiene las palabras que constituyen operandos del operador Distancia (~).
    \item public static bool[] operationsSwitch;
    Vector booleano que regula cuales operaciones se deben realizar según la query.
    \item public static bool QueryContainsOperators(string query);
    Método para determinar si la query emplea operadores de búsqueda.
    \item public static void SetMarkers(string query);
    Métodos que determina los marcadores (palabras operandos) en la query.
    \item private static string TakeWord(string input);
    Este método separa las palabras de los operadores.
    \item public static float ApplySearchOperators(DocumentVector doc);
    Método principal, se encarga de implementar los operadores de búsqueda
\end{enumerate}

\subsection{Moogle}
Moogle: La clase estática Moogle contiene los métodos que se ejecutan al iniciar el
proyecto y desarrollar, es el código que guía la ejecución del programa coordinando al
resto de las clases y métodos.
\emph{Elementos de la clase:}
\begin{enumerate}

\item public static Corpus corpus;
    La variable estática corpus permitirá a otras clases acceder a los elementos del
    corpus como el vocabulario, los idfs de los términos y el sistema de vectores.
\item public static void SetCorpus()
    Este método se invoca al ejecutar el proyecto. Establece el corpus y lo guarda en
    la variable estática 1.
\item public static SearchResult Query(string query)
    Se invoca al presionar el botón búsqueda en la aplicación. A través de este método
    se recibe la query del usuario, esta se transforma en un vector y se establece su
    similitud con cada documento, se genera el snippet y se forma el resultado final de
    la búsqueda
\end{enumerate}

\subsection{SearchItem}
SearchItem: Es el objeto mediante la cual se representa cada resultado de búsqueda,
conformada por el título del documento, el valor de Score y el snippet generado.

\emph{Elementos de la clase:}
\begin{enumerate}
    \item public string Title { get; private set; }
    \item public string Snippet { get; private set; }
    \item public double Score { get; private set; }
    \item public SearchItem(string title, string snippet, double score)
\end{enumerate}


\subsection{SearchItem}
SearchResult: Este objeto representa el resultado general de la búsqueda, es la
colección de resultados individuales (search items). Además dentro de esta clase se
implementa la funcionalidad Sugerencia a partir de métodos estáticos.

\emph{Elementos de la clase:}
\begin{enumerate}
    \item private SearchItem[] items;
    Arreglo de SearchItem, coleccion de los resultados individuales de la búsqueda
    \item public SearchResult(SearchItem[] items, string suggestion="")
    Constructor de la clase, genera el resultado de la búsqueda
    \item public string Suggestion { get; private set; }
    Sugerencia de la busqueda
    \item public SearchResult() : this(new SearchItem[0]) {}
    Constructor alternativo empleado cuando no se obtuvieron resultados en la búsqueda
    \item public IEnumerable<SearchItem> Items() => this.items;
    \item public int Count { get { return this.items.Length; } }
    Propiedad que devuelve la cantidad de elementos del resultado de la búsqueda
    \item public static string Suggestion(DocumentVector query)
    Sugerencia de la búsqueda
    \item private static string MostSimilarWordInCorpus(string term)
    Método para la sugerencia, encuentra,entre las palabras del corpus, la más similar 
    a otra palabra dada como argumento
    \item private static int EditDistance(string s1, string s2, int m ,int n)
    Calcula la distancia de Edición entre dos strings.

\section{Desarrollo}
\subsection{Estableciendo el Corpus}
Al ejecutarse el proyecto, mediante el paso anterior, automáticamente se llama al método public
static void SetCorpus() de la clase Moogle que se encarga de preprocesar la colección de docuementos y establecer el corpus.
El corpus es una instancia de la clase Corpus, el mismo recibe en su creación la ubicación de
la colección de documentos ContentPath, carga todos los archivos de este directorio que
contengan la el patrón “.txt”; inicializa el vocabulario de términos, el diccionario idfs, y el array
vectorList.
Mediante un ciclo for, para cada documento de la colección:
\begin{enumerate}
    \item Se lee el texto del mismo mediante el método File.ReadAllText() y a partir de este crea
    un DocumentVector.
    \begin{enumerate}
        \item Las instancias de la clase documentVector reciben un texto en su constructor.
        \item Procesa el texto, dividiéndolo en sus tokens(palabras) mediante el método
        Tokenize()
        \item Inicializa el set de palabras del documento \emph{words} y el diccionario
        docTermFrequencies.
        \item Calcula la frecuencia de todos los términos del documento.
    \end{enumerate}
    \item Asigna el nombre del archivo a la propiedad FileName del vector utilizando el método
    GetFileName().
    \item Llena el vocabulario de términos de la colección añadiendo los términos de cada
    documento. Al emplear un HashSet como estructura de datos para esta tarea se evita la
    repetición de palabras.
\end{enumerate}
    A continuación, se calcula el idf de cada palabra eliminando las stopwords. Se consideró stopword aquellas
    palabras que aparecen en más del 85\% de los documentos.

\begin{minted}{csharp}
    // Calculate IDFs and remove stop words
    var stopWordThreshold = Math.Log((double)100 / 85);
    foreach (var term in this.vocabulary)
    {
        var idf = Math.Log(documents.Length / idfs[term]);
        if (idf > stopWordThreshold)
            idfs[term] = idf;
        else
        {
            stopWords.Add(term);
            idfs.Remove(term);
            vocabulary.Remove(term);
        }
    }
\end{minted}

Se inicializan y calculan los vectores de pesos de los documentos con el método
SetWeightInCorpus(), luego se normalizan los vectores de pesos, de esta forma no
tenemos que realizar este paso durante la búsqueda , pues para hallar la similitud solamente restaría realizar el producto escalar.

\begin{center}
    \begin{equation}
    sim(A, B) = \frac{\overrightarrow{A} * \overrightarrow{B}}{||A|| * ||B||}  \rightarrow sim(A, B) = \overrightarrow{A'} * \overrightarrow{B'}
    \tag*{Producto Vectorial}
    \end{equation}
    \end{center}

Donde A' y B' son los vectores normalizados.
Luego se ejecuta la aplicación web en el navegador predetermiando donde el usuario introduce la query y presiona el botón
búsqueda que invoca al método Query() de la clase Moogle. Esta toma como argumento la
query introducida por el usuario y la transforma en un DocumentVector, calcula el array de pesos
y lo normaliza. Luego calcula los scores de cada documento a partir de la similitud de cosenos.

\subsection{Operadores de Búsqueda}
La estructura de control if nos ayuda a determinar si la query contiene o no operadores de
búsquedas.

\begin{minted}{csharp}
    if (SearchOperators.QueryContainsOperators(query))
    {
        SearchOperators.SetMarkers(query);
        corpus.RankDocumentsWithOperators();
    }

\end{minted}


El método booleano QueryContainsOperators(string query) crea un conjunto (HashSet) de los
caracteres de la query y lo interseca con la variable estática operators que contiene los
caracteres designados para los operadores. Si la interacción es vacía la query no contiene
operadores y devuelve false, en caso contrario la query contiene operadores y devuelve true.
Si la query contiene operadores se invoca el método SetMarkers (string query), este divide a la
query en palabras, creando un array y analiza mediante el ciclo for el primer carácter da cada
una de ellas en busca de operadores.
Se ejecuta la estructura de control Switch, con el default continue, es decir, si el primer carácter
no es ninguno de los operadores de búsqueda se pasa a la siguiente palabra de la query:

\begin{minted}{csharp}
    switch (firstChar)
    {
        case '!':
            nonExistenceMarkers = new HashSet<string>();
            nonExistenceMarkers.Add(TakeWord(term));
            operationsSwitch[0] = true;
            break;
        case '^':
            existenceMarkers = new HashSet<string>();
            existenceMarkers.Add(TakeWord(term));
            operationsSwitch[1] = true;
            break;
        case '*':
            importanceMarkers = new HashSet<(string word, int count)>();
            var word = TakeWord(term);
            var count = 1;
            for (var j = 1; j < term.Length; j++)
            {
                if (term[j] != '*') break;
                count++;
            }
            (string, int) pair = (word, count);
            importanceMarkers.Add(pair);
            operationsSwitch[2] = true;

            break;
        case '~':
            distanceMarkers = new HashSet<(string, string)>();
            var marker1 = TakeWord(queryTerms[i - 1]);
            var marker2 = TakeWord(term);
            (string, string) marker = (marker1, marker2);
            distanceMarkers.Add(marker);
            operationsSwitch[3] = true;
            break;  

        default:
            continue;
    }
\end{minted}


En caso de que primer carácter sea un operador: se inicializa la variable estática
correspondiente al operador ver \ref{sec: classes} y se la añade la palabra depurada
con el método TakeWord(), se “activa” el operationSwitch en la posición correspondiente.
El operationSwitch es un array de bools de cuatro elementos, uno para cada operador, cuando
el método SetMarkers encuentra un operador asigna true a la posición correspondiente al
operador en el array, esto nos ayudará a ¨recordar¨ que operadores debemos calcular más
adelante.
El operador * se pude emplear cualquier cantidad de veces consecutivas, por eso en su case
empleamos un ciclo for y una variable auxiliar count para contar la cantidad de veces que se
empleó.
Al ser binario, el operador de distancia requiere dos operandos. En este case se toma la palabra
que lo contine y la palabra anterior, se unen en una tupla y se guardan los pares de palabras
en tuplas en la variable estática distanceMarkers.
Luego se procede aplicar el método RankDocumentsWithOperators(queryVector):
Empleando un ciclo for, para cada vector – documento, cuyos scores han sido calculado
previamente según su similitud de cosenos con la query, se le aplica un modificador que
consiste en multiplicar el score por un factor scoreModifier que se calcula con el método
ApplySearchOperators (DocumentVector doc):
Este método “decide” (if) que operadores se deben calcular “preguntándole” al operationSwitch;
calcula y devuelve el scoreModifier que es la suma del aporte de cada uno de los operadores
aplicados (ver Introducción). Para ello invoca los métodos auxilares ApplyXOp
(DocumentVector doc) donde \emph{X} es el operador correspondiente.
    %Piece of Code

\subsection{Resultados:}    
La propiedad Ranking del corpus nos devuelve un arreglo, ordenado de forma decreciente, con
todos los vectores-documento con un score mayor que 0, todos ellos se mostraran como
resultados de la búsqueda, en el orden mencionado, transformándolos antes a SearchItems.
Para la generación del Snippet se llama al método Snippet.GetSnippet(queryVector, docVector),
este determina la palabra de mayor importancia(peso) de la query que aparece en el
documento,
static string MostRelevantWord(DocumentVector queryVector, DocumentVector docVector)
La ubica dentro del texto
public static int GetIndexOf(string word)
y extrae una ventana del texto alrededor de dicha palabra con el metodo static string GetTextPieceAround(int index) 

\section{}

\end{document}